%%
%% Automatically generated file from DocOnce source
%% (https://github.com/hplgit/doconce/)
%%
%%


%-------------------- begin preamble ----------------------

\documentclass[%
oneside,                 % oneside: electronic viewing, twoside: printing
final,                   % draft: marks overfull hboxes, figures with paths
10pt]{article}

\listfiles               %  print all files needed to compile this document

\usepackage{relsize,makeidx,color,setspace,amsmath,amsfonts,amssymb}
\usepackage[table]{xcolor}
\usepackage{bm,ltablex,microtype}

\usepackage[pdftex]{graphicx}

\usepackage[T1]{fontenc}
%\usepackage[latin1]{inputenc}
\usepackage{ucs}
\usepackage[utf8x]{inputenc}

\usepackage{lmodern}         % Latin Modern fonts derived from Computer Modern

% Hyperlinks in PDF:
\definecolor{linkcolor}{rgb}{0,0,0.4}
\usepackage{hyperref}
\hypersetup{
    breaklinks=true,
    colorlinks=true,
    linkcolor=linkcolor,
    urlcolor=linkcolor,
    citecolor=black,
    filecolor=black,
    %filecolor=blue,
    pdfmenubar=true,
    pdftoolbar=true,
    bookmarksdepth=3   % Uncomment (and tweak) for PDF bookmarks with more levels than the TOC
    }
%\hyperbaseurl{}   % hyperlinks are relative to this root

\setcounter{tocdepth}{2}  % levels in table of contents

% prevent orhpans and widows
\clubpenalty = 10000
\widowpenalty = 10000

% --- end of standard preamble for documents ---


% insert custom LaTeX commands...

\raggedbottom
\makeindex
\usepackage[totoc]{idxlayout}   % for index in the toc
\usepackage[nottoc]{tocbibind}  % for references/bibliography in the toc

%-------------------- end preamble ----------------------

\begin{document}

% matching end for #ifdef PREAMBLE

\newcommand{\exercisesection}[1]{\subsection*{#1}}


% ------------------- main content ----------------------



% ----------------- title -------------------------

\thispagestyle{empty}

\begin{center}
{\LARGE\bf
\begin{spacing}{1.25}
Computating in Physics courses  at the Physics and Astronomy Department, Michigan State University
\end{spacing}
}
\end{center}

% ----------------- author(s) -------------------------

\begin{center}
{\bf Danny Caballero, Sean Couch, Wade Fisher, Connor Glosser, Morten Hjorth-Jensen, Claire Kopenhafer, Brian O'Shea, and Carlo Piermarocchi${}^{}$} \\ [0mm]
\end{center}

\begin{center}
% List of all institutions:
\end{center}
    
% ----------------- end author(s) -------------------------

% --- begin date ---
\begin{center}
Mar 27, 2017
\end{center}
% --- end date ---

\vspace{1cm}


\subsection*{Introduction: Scientific and educational motivation}

Numerical simulations of various systems in science are central to our
basic understanding of nature and technology.
The increase in computational power,
improved algorithms for solving problems in science as well as access
to high-performance facilities, allow researchers to study
complicated systems across many length and energy scales. Applications
span from studying quantum physical systems in nanotechnology and the
characteristics of new materials or subatomic physics at its smallest
length scale, to simulating galaxies and the evolution of the universe.
In between, simulations are key to understanding
cancer treatment and how the brain works,
predicting climate changes and this week's weather,
simulating natural disasters, semi-conductor devices,
quantum computers, as well as assessing risk in the insurance and
financial industry.


\paragraph{Computing competence.}
Computing means solving scientific problems using computers. It covers
numerical as well as symbolic computing. Computing is also about
developing an understanding of the scientific process by enhancing
algorithmic thinking when solving problems.  Computing competence has
always been a central part of education in the sciences and engineering disciplines.

On the part of students, this competence involves being able to:

\begin{itemize}
\item understand how algorithms are used to solve mathematical problems,

\item derive, verify, and implement algorithms,

\item understand what can go wrong with algorithms,

\item use these algorithms to construct reproducible scientific outcomes and to engage in science in ethical ways, and

\item think algorithmically for the purposes of gaining deeper insights about scientific problems.
\end{itemize}

\noindent
All these elements are central for maturing and gaining a better understanding of the modern scientific process \emph{per se}.

The power of the scientific method lies in identifying a given problem
as a special case of an abstract class of problems, identifying
general solution methods for this class of problems, and applying a
general method to the specific problem (applying means, in the case of
computing, calculations by pen and paper, symbolic computing, or
numerical computing by ready-made and/or self-written software). This
generic view on problems and methods is particularly important for
understanding how to apply available, generic software to solve a
particular problem.

Computing competence represents a central element
in scientific problem solving, from basic education and research to
essentially almost all advanced problems in modern
societies. Computing competence is simply central to further
progress. It enlarges the body of tools available to students and
scientists beyond classical tools and allows for a more generic
handling of problems. Focusing on algorithmic aspects results in
deeper insights about scientific problems.





\paragraph{Why should basic university education undergo a shift towards modern computing?}
\begin{itemize}
\item Algorithms involving pen and paper are traditionally aimed at what we often refer to as continuous models.

\item Application of computers calls for approximate discrete models.

\item Much of the development of methods for continuous models are now being replaced by methods  for discrete models in science and industry, simply because much larger classes of problems can be addressed with discrete models, often also by simpler and more generic methodologies.
\end{itemize}

\noindent
However, verification of algorithms and understanding their limitations requires much of the classical knowledge about continuous models.

So, why should basic university education undergo a shift towards modern computing?

The impact of the computer on mathematics and science is tremendous: science and industry now rely on solving mathematical problems through computing.
\begin{itemize}
\item Computing can increase the relevance in education by solving more realistic problems earlier.

\item Computing through programming can be excellent training of creativity.

\item Computing can enhance the understanding of abstractions and generalization.

\item Computing can decrease the need for special tricks and tedious algebra, and shifts the focus to problem definition, visualization, and "what if" discussions.
\end{itemize}

\noindent
The result is a deeper understanding of mathematical modeling and the scientific method (we hope, and here our physics education research group can play a central role in promoting this).
Not only is computing via programming a very powerful tool, it can also be a great pedagogical aid.

For the mathematical training, there is one major new component among the arguments above: understanding abstractions and generalization. While many of the classical methods developed for continuous models are specialized for a particular problem or a narrow class of problems, computing-based algorithms are often developed for problems in a generic form and hence applicable to a large problem class.


Computing competence represents a central element in scientific problem solving, from basic education and research to essentially almost all advanced problems in modern societies. Computing competence is simply central to further progress. It enlarges the body of tools available to students and scientists beyond classical tools and allows for a more generic handling of problems. Focusing on algorithmic aspects results in deeper insights about scientific problems.

Moreover, today's projects in science and industry tend to involve larger teams. Tools for reliable collaboration must therefore be mastered (e.g., version control systems, automated computer experiments for reproducibility, software and method documentation).


\subsection*{General learning outcomes for computing competence}

Below, we articulate high-level learning outcomes that we expect students to develop through comprehensive and coordinated instruction in numerical methods over the course of their bachelor's program at Michigan State. These learning outcomes are different from specific learning goals in that the former reference the end state that we aim for students to achieve. The latter references the specific knowledge, tools, and practices with which students should engage and discusses how we expect them to participate in that work. We reserve the discussion of specific learning goals to individual course experiences (e.g., Electrostatics).

\paragraph{Learning outcomes for numerical algorithms.}
Numerical algorithms form the basis for solving science and engineering problems with computers. An understanding of algorithms does not itself serve as an understanding on computing, but it is a necessary step along the path. Through comprehensive and coordinated instruction, we aim for students to have developed:

\begin{itemize}
\item A deep understanding of the most fundamental algorithms for linear algebra, ordinary and partial differential equations, optimization, and statistical uncertainty quantification

\item A working knowledge of advanced algorithms and how they can be accessed in available software

\item A working knowledge of high-performance computing elements including memory usage, vectorized and parallel algorithms

\item A deep understanding of approximation errors and how they can present themselves in different problems

\item The ability to apply fundamental and advanced algorithms to classical model problems as well as real-world problems as well to assess the uncertainty of their results
\end{itemize}

\noindent
\paragraph{Learning outcomes for symbolic computing.}
Symbolic computing is a helpful tool for addressing certain classes of problems where a functional representation of the solution (or part of the solution) is needed. Through engaging with symbolic computing platforms, we aim for students to have developed:

\begin{itemize}
\item A working knowledge of at least one computer algebra system (CAS)

\item The ability to apply a CAS to perform classical mathematics including calculus, linear algebra, differential equations

\item The ability to verify the results produced by the CAS using some other means
\end{itemize}

\noindent
\paragraph{Learning outcomes for programming.}
Programming is a necessary aspect of learning computing for science and engineering. The specific languages and/or environments that students learn are less important than the nature of that learning (i.e., learning programming for the purposes of solving science problems). By numerically solving science problems, we expect students to have developed:

\begin{itemize}
\item An understanding of programming in a high-level language (e.g., MATLAB, Python, R).

\item An understanding of programming in a compiled language (e.g., Fortran, C, C++).

\item The ability to to implement and apply numerical algorithms in reusable software that acknowledges the generic nature of the mathematical algorithms.

\item A working knowledge of basic software engineering elements including functions, classes, modules/libraries, testing procedures and frameworks, scripting for automated and reproducible experiments, documentation tools, and version control systems (e.g., GitHub).

\item An understanding of debugging software, e.g., as part of implementing comprehensive tests.
\end{itemize}

\noindent
\paragraph{Learning outcomes for mathematical modeling.}
Preparing a problem to be solved numerically (i.e., modeling) is a critical step in making progress towards an eventual solution. By providing opportunities for students engage in modeling, we aim for them to develop:

\begin{itemize}
\item The ability to solve real problems from applied sciences by:
\begin{itemize}

  \item Deriving computational models from basic principles in physics and articulating the underlying assumptions in those models,

  \item Constructing models with dimensionless forms to reduce and simplify input data, and

  \item Interpreting the model's dimensionless parameters to increase their understanding of the model and its predictions
\end{itemize}

\noindent
\end{itemize}

\noindent
\paragraph{Learning outcomes for verification.}
Verifying the model and the resulting outcomes it produces are essential elements to generating confidence in the model itself. Moreover, such verifications provide evidence that the work is reproducible. By engaging in verification practices, we aim for for students to develop:

\begin{itemize}
\item An understanding of how to program testing procedures

\item A deep knowledge of testing/verification methods including the use of:
\begin{itemize}

  \item Exact solutions of numerical models

  \item Method of manufactured solutions (i.e., choose solution and fit a problem)

  \item Classical analytical solutions including asymptotic solutions

  \item Computed asymptotic approximation errors (i.e., convergence rates)

  \item Step-wise construction of tests to aid debugging.
\end{itemize}

\noindent
\end{itemize}

\noindent
\paragraph{Learning outcomes for presentation of results.}
The results of a computation need to be communicated in some format (i.e., through figures, posters, talks, and other forms of written and oral communication). Computation affords the experience of presenting original results quite readily. Through their engagement with presentations for their findings, we aim for students to develop:

\begin{itemize}
\item The ability to make use of different visualization techniques for different types of computed data

\item The ability to present computed results in scientific reports and oral presentations effectively

\item A working knowledge of the norms and practices for scientific presentations in various formats (i.e., figures, posters, talks, and written reports)
\end{itemize}

\noindent
\subsection*{Specific algorithms and computational skills}

The above learning goals and outcomes are of a more generic character. What follows here are specific
algorithms that occur frequently in scientific problems. The implementation of these algorithms in various physics courses, together with problem and project solving, is a way to implement large fractions of the above learning goals. we reserve the coupling of the the broad learning goals above to the algorithms articulated below to our discussion of specific course (or topical) learning goals.

We list also tools that are important in developing
numerical projects. These tools allow students to develop a better understanding of the scientific process. In addition, use of the algorithms can facilitate instruction in an ethical approach to science.

The algorithms and tools listed here can be intergated in different ways depending on the specific learning goals for the course or topic. Furthermore, several of these algorithms can be used and, then, revisited in the various courses.

\paragraph{Central algorithms.}
The following mathematical formulations of problems from the physical sciences play a prominent role and should be reflected in how we teach physics:
\begin{itemize}
 \item Ordinary differential equations
\begin{enumerate}

  \item Euler, modified Euler, Verlet and Runge-Kutta methods with applications to problems in electromagnetism, methods for theoretical physics, quantum mechanics and mechanics.

\end{enumerate}

\noindent
 \item Partial differential equations
\begin{enumerate}

  \item Diffusion in one and two dimensions (statistical physics), wave equation in one and two dimensions (mechanics, electromagnetism, quantum mechanics, methods for theoretical physics) and Laplace's and Poisson's equations (electromagnetism).

\end{enumerate}

\noindent
 \item Numerical integration
\begin{enumerate}

  \item Trapezoidal and Simpson's rule and Monte Carlo integration. Applications in statistical physics, methods of theoretical physics, electromagnetism and quantum mechanics.

\end{enumerate}

\noindent
 \item Statistical analysis, random numbers, random walks, probability distributions, Monte Carlo integration and Metropolis algorithm. Applications to statistical physics and laboratory courses.

 \item Linear Algebra and eigenvalue problems.
\begin{enumerate}

  \item Gaussian elimination, LU-decomposition, eigenvalue solvers, and iterative methods like  Jacobi or Gauss-Seidel for systems of linear equations. Important for several courses, classical mechanics, methods of theoretical physics, electromagnetism and quantum mechanics.

\end{enumerate}

\noindent
 \item Signal processing
\begin{enumerate}

  \item Discrete (fast) Fourier transforms, Lagrange/spline/Fourier interpolation, numeric convolutions {\&} circulant matrices, filtering. Applications in electromagnetics, quantum mechanics, and experimental physics (data acquisition)

\end{enumerate}

\noindent
 \item Root finding techniques, used in methods for theoretical physics, quantum mechanics, electromagnetism and mechanics.
\end{itemize}

\noindent
In order to achieve a proper pedagogical introduction of these algorithms, it is important that students and teachers alike see how these algorithms are used to solve a variety of physics problems. The same algorithm, for example the solution of a second-order differential equation, can be used to solve the equations for the classical pendulum in a mechanics course or the (with a suitable change of variables) equations for a coupled RLC circuit in the electromagnetism course. Similarly, if students develop a program for studies of celestial bodies in the mechanics course, many of the elements of such a program can be reused in a molecular dynamics calculation in a course on statistical physics and thermal physics. The two-point boundary value problem for a buckling beam
(discretized as an eigenvalue problem) can be reused in quantum mechanical studies of interacting electrons in oscillator traps, or just to study a particle in a box potential with varying depth and extension.

In order to aid the introduction of computational exercises and projects, we will need to develop educational resources for this. The \href{{http://www.compadre.org/picup/}}{PICUP project},  Partnership for Integration of Computation into Undergraduate Physics, develops \href{{http://www.compadre.org/PICUP/resources/}}{resources for teachers and students on the integration of computational  material}.   We strongly recommend these resources.  Physics is an old discipline, with a large wealth of established analytical exercises and projects. In fields like mechanics, we have centuries of pedagogical developments, with a strong emphasis on developing analytical skills. The majority of physics teachers are well familiar with this and in order to see how computing can enlarge this body of exercises and projects, and hopefully add additional insights to the physics behind various phenomena, we find it important to develop a large body of computational examples.

As part of this proposal, we are in the process of  developing several examples of problems and projects that can be included in our undergraduate courses in physics. You can click on the following links (ipython notebooks or PDF formats)
\begin{itemize}
\item Mechanics
\begin{enumerate}

  \item \href{{https://github.com/CompPhysics/PAComputationsMSU/tree/master/doc/pub/Mechanics/ipynb}}{Ipython notebook}

  \item \href{{https://github.com/CompPhysics/PAComputationsMSU/tree/master/doc/pub/Mechanics/pdf/Mechanics-minted.pdf}}{Standard PDF file}

\end{enumerate}

\noindent
\item Quantum mechanics
\begin{enumerate}

  \item \href{{https://github.com/CompPhysics/PAComputationsMSU/tree/master/doc/pub/Quantum/ipynb}}{Ipython notebook}

  \item \href{{https://github.com/CompPhysics/PAComputationsMSU/tree/master/doc/pub/Quantum/pdf/Quantum-minted.pdf}}{Standard PDF file}

\end{enumerate}

\noindent
\item Electromagnetism
\begin{enumerate}

  \item \href{{https://github.com/CompPhysics/PAComputationsMSU/tree/master/doc/pub/Elmag/ipynb}}{Ipython notebook}

  \item \href{{https://github.com/CompPhysics/PAComputationsMSU/tree/master/doc/pub/Elmag/pdf/Elmag-minted.pdf}}{Standard PDF file}

\end{enumerate}

\noindent
\item Statistical and thermal physics
\begin{enumerate}

  \item \href{{https://github.com/CompPhysics/PAComputationsMSU/tree/master/doc/pub/StatPhys/ipynb}}{Ipython notebook}

  \item \href{{https://github.com/CompPhysics/PAComputationsMSU/tree/master/doc/pub/StatPhys/pdf/StatPhys-minted.pdf}}{Standard PDF file}

\end{enumerate}

\noindent
\item Methods in Theoretical Physics (not yet ready)
\begin{enumerate}

  \item \href{{https://github.com/CompPhysics/PAComputationsMSU/tree/master/doc/pub/}}{Ipython notebook}

  \item \href{{https://github.com/CompPhysics/PAComputationsMSU/tree/master/doc/pub/}}{Standard PDF file}

\end{enumerate}

\noindent
\item Physics laboratory (not yet ready)
\begin{enumerate}

  \item \href{{https://github.com/CompPhysics/PAComputationsMSU/tree/master/doc/pub/}}{Ipython notebook}

  \item \href{{https://github.com/CompPhysics/PAComputationsMSU/tree/master/doc/pub/}}{Standard PDF file}
\end{enumerate}

\noindent
\end{itemize}

\noindent
We hope these examples can aid physics teachers in their usage of computing in physics courses. Furthermore, via a consensus and collectively driven approach, the hope is that this body of examples can be continuously enlarged.

\paragraph{Central tools and programming languages.}
We will strongly recommend that Python is used as the high-level programming language for all the courses proposed below. Other high-level environments like Mathematica and Matlab can also be presented and offered as special courses like PHY102 - Physics Computations I. This means that students can apply their knowledge from CMSE 201, which makes use of Python, and extend their computational knowledge in various physics classes. We recommend strongly that the following tools are used
\begin{enumerate}
\item \href{{http://jupyter.org/}}{Jupyter and ipython notebook}.

\item Version control software like \href{{https://git-scm.com/}}{git} and repositories like \href{{https://github.com/}}{GitHub}

\item Other typsetting tools like {\LaTeX}.

\item Unit tests and using existing tools for unit tests. \href{{https://docs.python.org/2/library/unittest.html}}{Python has extensive tools for this}
\end{enumerate}

\noindent
The notebooks can be used to hand in exercises and projects. They can provide the students with experience in presenting their work in the form of scientific/technical reports.

Version control software allows teachers to bring in reproducibility of science as well as enhancing
collaborative efforts among students. Using version control can also be used to help students present benchmark results, allowing others to verify their results.

Unit testing is a central element in the development of numerical projects, from microtests of code fragments, to intermediate merging of functions to final test of the correctness of a code.

\subsection*{Suggested learning goals for specific physics courses}

For a major in physics degree at Michigan State University, the course \href{{https://cmse.msu.edu/academics/undergraduate-program/undergraduate-courses/cmse-201-introduction-to-computational-modeling/}}{CMSE 201 Introduction to Computational Modeling} is compulsory and it lays the foundation for the use of computational exercises and projects in various physics courses. Based on this course, and the various mathematics courses included in a Physics degree, there is a unique possibility to incorporate computinational exercises and projects in various physics courses, without taking away the attention from the basic physics topics to be covered. 

What follows below is a suggested listed of possible learning outcomes. These suggestions are to be viewed as inputs to the ongoing discussions. The list of possible outcomes can be reduced or enlarged. 

% 
% Danny: I have only constructed learning goals for PHY 481. I intend these
% to provide a example of what our goals might look like. They are debatable
% and not set in stone, but what they do is provide us with a starting point
% for discussion and an example of what measurable (i.e., assessable)
% learning goals might look like.
% 

We propose that the following courses aim to integrate the above learning outcomes and goals:


\paragraph{Classical Mechanics 1 PHY321.}
After completing Classical Mechanics 1 PHY321, students should be able to:

\begin{enumerate}
  \item Here one could focus on ordinary differential equations like modified Euler, Verlet and perhaps RK methods. There are several interesting problems that can deepen the understanding of newtonian problems, from modeling planetary motion to models for friction and earthquake simulations. With not too advanced programs one can easily bring in actual research problems (friction is a good example).  The classical pendulum is also a good case to illustrate. The code can easily be reused for studies of for example an RLC circuit in electromagnetism. The equations for a buckling beam lead to an eigenvalue problem which can reused in the quantum mechanics course.
\end{enumerate}

\noindent
\paragraph{Thermal and Statistical Physics PHY410.}
After completing Thermal and Statistical Physics PHY410, students should be able to:
\begin{itemize}
  \item use central probability distributions and their relation to various expectation values

  \item simulate and visualize central probability distributions like the uniform distributions, the exponential distribution and the normal (Gaussian distribution)

  \item use concept from statistics to understand central ensembles like the microcanonical and the canonical ensembles

  \item simulate Markov processes and understand the links with the process of diffusion (Fick's and Fourier's laws) and the concept of most likely states

  \item understand how to simulate stochastic variables using random number generators

  \item understand central algorithms like the Metropolis algorithm to simulate systems in statistical physics

  \item understand the physics of various phases and phase transitions 

  \item study systems like ideal gas and ideal crystals analytically and numerically

  \item understand the link between various ensembles; both mathematical and physical links

  \item be able to simulate phase transitions via models like the Ising class of models

  \item be able to perform molecular dynamics calculations using the velocity Verlet algorithm and simulate phase transitions and visualize and analyze results using realistic interactions.  
\end{itemize}

\noindent
\paragraph{Methods in Theoretical Physics PHY415.}
After completing Methods in Theoretical Physics PHY415, students should be able to:
\begin{itemize}
  \item Understand how to discretize differential equations and understand the mathematical truncation errors

  \item Understand errors in mathematical algorithms

  \item be able to rewrite differential equations using methods from linear algebra

  \item know important algorithms for solving eigenvalue problems

  \item be able to solve initial value and boundary value problems analytically and numerically

  \item know central algorithms for solving eigenvalue problems

  \item Solve differential equations numerically and compare with analytical solutions

  \item understand  important orthogonal orthogonal polynomials like Legendre, Hermite and Laguerre. Be able to set up their recursion relations and visualize the polynomials. 

  \item Understand Fourier transforms and algorithms like Fast Fourier transform

  \item Know tools to analyze time series

  \item more...
\end{itemize}

\noindent
Many of these algorithms can be discussed and used in the other courses discussed here.

\paragraph{Advanced Laboratory course in Physics PHY451.}
After completing Advanced Laboratory course in Physics PHY451, students should be able to:

\begin{enumerate}
  \item Data analysis plays a central role, natural to have algorithms on data fitting, computations  of mean values, variance and standard deviations, covariance, famous distributions. Many of these topics find also their natural place in a course on statistical physics like PHY410.
\end{enumerate}

\noindent
\paragraph{Quantum mechanics 1  PHY 471.}
After completing Quantum mechanics 1  PHY 471, students should be able to:
\begin{itemize}
  \item be able to visualize the solutions of quantum mechanical problems, both stationary and time-dependent problems

  \item be able to scale the equations properly and understand the meaning of natural length scales, from the Bohr radius to simple harmonic oscillator problems with frequency dependent length scale. The same scaling procedure is  used to derive the analytical solutions for several single-particle problems. 

  \item use numerical methods for solving a large variety of one-dimensional differential equations with two-point boundary value problems. For many cases one can compare directly with standard analytical  solutions like the hydrogen-like problems or the harmonic oscillator.

  \item Verification of numerical solutions with analytical results.

  \item be able to rewrite Schroedinger's equation as an eigenvalue problem and use numerical eigenvalue methods for computing a single-particle confined in a one-dimensional infinite potential and compare with analytical results. This problem is the same as the eigenvalue problem of a buckling beam, which can be used the in the mechanics and mathematical methods course. 

  \item use the the same eigenvalue solvers to study a single particle confined to move in a potential well with a finite depth and extension. Study both bound and unbound states and explore the numerical solutions as functions of the potential depth and the extension of the potential. 

  \item The same codes can be used to solve the hydrogen atom and the one-dimensional harmonic oscillator. This part allows for comparison with analytical results. 

  \item Visualize the probability distributions for electrons (or other one-particle problems) confined to move in hydrogen-like and harmonic oscillator like problems. Study the probability distributions for ground and excited states. Discuss unbound states with a finite potential well.

  \item Use the same codes to study double-well potentials. These are problems of great interest in solid state physics.

  \item Rewrite a two-electron (or two-particle problem) problem in terms of the relative and center-of-mass motion and study the role of repulsive Coulomb forces) for electrons (or other fermions) trapped 

  \item Visualize and compute tunneling phenomena for various potentials.

  \item Introduce the  variational principle and introduce variational Monte Carlo methods to study one-particle problems and compare these with analytical results and numerical results from differental equation solvers.  The Metropolis algorithm discussed in Statistical physics can be reused here. Gives the students a further understanding of statistics related topics, including random number generators, probability distributions, mean values and standard deviations. These issues would also be discussed in PHY415. The students will then see central algorithms being used in different physics settings.
\end{itemize}

\noindent
Many of these topics can be included and extended upon in PHY472. 
\paragraph{Electromagnetism 1 PHY481.}
After completing Electromagnetism 1 PHY481, students should be able to:
\begin{itemize}
  \item use symbolic computing tools to determine the gradient of various scalar fields

  \item use symbolic computing to determine the divergence and curl of various vector fields

  \item represent the vector (e.g., electric) field visually using vector plots and stream plots

  \item represent a 2D scalar (i.e., potential) field visually using 2D contour plots and 3D surface plots

  \item apply motion prediction algorithms Euler, Verlet, and Runge-Kutta to model the motion of charged particles in electric and magnetic fields

  \item compare the quality of simulations (i.e., number of iterations, step size, and error control) of charged particle motion that use different motion prediction algorithms

  \item apply Coulomb's law iteratively to determine the electric field produced by a given charge distribution

  \item apply Biot-Savart's law iteratively to determine the magnetic field produced by a given current distribution

  \item explain how the application of superposition iteratively gives rise to approximate field solutions

  \item explain how the simple relaxation algorithm works (i.e., iteratively averaging neighboring points) and how it is derived from the properties of the solutions to Laplace's equation

  \item apply this simple relaxation method to find the potential for 1D and 2D electrostatic situations where Laplace's equation is satisfied

  \item explain how to use the finite-differencing to recast Poisson's equation into a discrete formulation and how the resulting discretized form compares with the simple relaxation method (i.e., iteratively averaging neighboring points)

  \item apply the Jacobi and Gauss-Seidel methods to solve 2D Laplace and Poisson problems including graphing the results in three dimensions

  \item explain the differences between the Jacobi and Gauss-Seidel methods and how these methods are connected to the derivation using finite differencing

  \item Compare the quality of the simulations (i.e., number of iterations, step size, and error control) that employ the Jacobi method and the Gauss-Seidel method

  \item See \href{{https://dannycab.github.io/phy481msu/learning_goals.html}}{the detailed learning outcomes for PHY 481} for more information.
\end{itemize}

\noindent
Many of these topics can be included and extended upon in PHY482. 

% 
% Danny: I haven't yet reviewed the material below in detail because I wanted
% us to spend time discussing the learning outcomes. These learning outcomes
% are connected to the work that the Physics Education Research Lab would do
% as we help to develop assessments and evaluations to share with the faculty
% 

\subsection*{Physics Education research and computing in science education}

The introduction of computational elements in the various courses should be stronly integrated with ongoing research on physics education.

The Physics and Astronomy department at MSU is in a unique position due to its strong research group in physics education, the \href{{http://www.pa.msu.edu/research/physics-education-lab}}{PERL group}.
This means that it is possible to
develop research motivated curricular changes.  There are many interesting challenges here, like
which are the main obstacles when transferring from a classical pen and paper approach to actually have a working program which solves the same (and more general problems). What is a good progression in presenting numerical topics in physics courses? Are there specific mathematical skills we would like our students to have? How do we integrate student-active teaching, how do we develop and test various assessment methods?



\subsection*{Links with CSME and Mathematics courses}

We need to figure out, beyond CSE 201, if computational elements, or some of the abovementioned algorithms are included in the mathematics courses our physics students take.


\subsection*{Advanced computational physics courses}

Towards the end of undergraduate studies it is useful to offer a course which focuses on more advanced algorithms and presents compiled languages like C++ and Fortran, languages our students will meet in actual research.
Furthemore, such a course, like the present \href{{https://github.com/CompPhysics/ComputationalPhysicsMSU}}{PHY480 Computational Physics} offers as well more advanced projects which train the students in actual research, developing more complicated programs and working on larger projects.
Such a course could cover
\begin{itemize}
  \item C++ and/or Fortran programming

  \item Numerical derivation and integration

  \item Random numbers and Monte Carlo integration

  \item Monte Carlo methods in statistical physics

  \item Quantum Monte Carlo methods

  \item Linear algebra and eigenvalue problems

  \item Non-linear equations and roots of polynomials

  \item Ordinary differential equations

  \item Partial differential equations

  \item Parallelization of codes

  \item High-performance computing aspects and optimization of codes
\end{itemize}

\noindent
Then we should consider more advanced (at the graduate level) courses which could cover specialized topics.
These could be (presently some of this is covered by PHY905, sections 002 and 003)
\begin{enumerate}
  \item Computational quantum mechanics, spanning from Monte Carlo methods to other methods used in atomic, molecular, nuclear, condensed matter and high-energy physics

  \item Computational Statistical mechanics, covering topics spanning from Molecular dynamics to studies of phase transitions, highly relevant for several subdisciplines

  \item Computational Astrophysics and Astronomy

  \item Data analysis and machine learning tailored to physics problems. This course can be jointly taught with the new department of Computational Science.

  \item High-performance computing

  \item Other courses
\end{enumerate}

\noindent
\subsection*{A possible Physics major in computational physics and a possible MSc in Computational Physics}

The above advanced courses can form the basis for a MSc in computational physics, as well as, together with 
CSME courses, a major in computational physics. Need to elaborate a meaningful path here. 


% 
% Danny: I found the material written below as either being stated in other
% parts of the document in some way or having too much detail for this
% document. I think we could move it back or edit it if folks felt strongly,
% but I felt quite simply that the document wasn't losing anything by leaving
% it out
% 


% ------------------- end of main content ---------------

\end{document}

